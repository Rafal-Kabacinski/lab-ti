\documentclass[10pt,a4paper]{article}
\usepackage[utf8]{inputenc}
\usepackage[T1]{fontenc}
\usepackage[polish]{babel}
\usepackage{graphicx}
\usepackage{polski} 
\usepackage{listings}

\title{Przykładowy dokument w \LaTeX u}
\author{Rafał Kabaciński}
\begin{document}
	
	\maketitle
	
	\tableofcontents
	
	\section{\LaTeX\ w praktyce}
	Ten dokument służy jako przykład jak łatwo można napisać pierwszy plik w \LaTeX u. 
	
	Każdy akapit, poza pierwszym, domyślnie rozpoczyna się wcięciem.
	Nowy akapit można rozpocząć, oddzielając go od poprzedniego dwoma znakami \emph{enter} lub \verb|\\|.
	
	
	
	\LaTeX\ ignoruje        wiele powtarzających się ,,białych znaków''. Tak więc napisanie: \verb|słowo słowo| lub \verb|słowo          słowo| da taki sam efekt. Tak samo zignorowane zostanie więcej niż dwa znaki \emph{enter}.
	
	Zalety \LaTeX a:
	\begin{itemize}
		\item ułatwi pisanie spójnych stylistycznie długich dokumentów,
		\item tworzenie list elementów, takich jak spis treści czy rysunków jest w nim łatwe,
		\item można w nim pisać listy wypunktowane.
	\end{itemize}
	
	\section{Formuły matematyczne w \LaTeX u}
	
	Wzór na odwrotną transformatę Fouriera:
	\begin{equation}
		f(x) = \lim \limits_{T\rightarrow + \infty} \int \limits_{-T}^{T} \hat f (\xi) e^{2\pi i x \xi} d\xi
	\end{equation}
	
	Wzór na macierz rotacji w przestrzeni:
	$$
	Rot_X(\alpha) = \left[
	\begin{array}{ccc}
	1 & 0 & 0 \\
	0 & \cos \alpha & -\sin \alpha \\
	0 & \sin \alpha & \cos \alpha 
	\end{array}\right]
	$$
	
	\section{Tabele i rysunki}
	
	Przykładowa tabela: \ref{tab} i rysunek: \ref{rys}.
	
	\begin{table}[h!]
		\centering
		\caption{Przykładowa tabela}
		\label{tab}
		\begin{tabular}{c|c|c|c}
			a & b & c & d \\ \hline
			1 & 2 & 3 & 4
		\end{tabular}
	\end{table}

	\begin{figure}[h!]
		\centering
		\includegraphics[width=0.5\textwidth]{Rys/pp-putlogopelne.png}
		\caption{Przykładowy rysunek}
		\label{rys}
	\end{figure}

\section{Kod źródłowy w dokumentach}

	\begin{lstlisting}[language=C++]
		for(int i =0;1<10;i++)
		{ 
			cout<< "i = " << i << endl;
		}
	\end{lstlisting}
	
	
\end{document}